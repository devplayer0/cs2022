\documentclass[a4paper]{article}

\usepackage[a4paper,margin=2cm]{geometry}
\usepackage{amsmath}
\usepackage{graphicx}
\usepackage[table]{xcolor}
\usepackage{tikz}
\usepackage{minted}
\usepackage[clock]{ifsym}
\usepackage{subcaption} % subfigures
\usepackage{hyperref} % links in table of contents
\usepackage[strings]{underscore}

\usetikzlibrary{calc,positioning,shapes,arrows.meta,decorations.pathreplacing}
\graphicspath{ {./graphics/} }

\hypersetup{
	colorlinks,
	citecolor=black,
	filecolor=black,
	linkcolor=black,
	urlcolor=black
}
\numberwithin{figure}{section}
\numberwithin{table}{section}
\renewcommand{\arraystretch}{1.5}

\newcommand{\mi}{\mintinline}
\newcommand{\NA}{---}

\title{CS2022 - Project 2}
\date{2019-04-14}
\author{\url{https://git.nul.ie/dev/cs2022}\\\url{https://github.com/devplayer0/cs2022}\\Jack O'Sullivan\\\href{mailto:osullj19@tcd.ie}{osullj19@tcd.ie}\\17331147}

\begin{document}
\maketitle
\tableofcontents
\pagenumbering{gobble}

\newpage
\pagenumbering{arabic}
\section{Introduction}
The goal of this assignment was to complete a functional microprogrammed processor, implementing a number of intructions.

\section{Testbench results}
This section details results of the testbenches for the components created in project 2.

\subsection{\mi{c}{memory}}
\begin{figure}[h!]
	\centering
	\includegraphics[width=\textwidth]{memory_tb}
	\caption{\mi{c}{memory} testbench results}
	\label{fig:memory}
\end{figure}

Figure~\ref{fig:memory} shows the simulation results of the 512x 16-bit word main memory.
\begin{itemize}
	\item After a short delay, setting the input address to \mi{c}{0} produces \mi{c}{0x203} (the first 
		instruction in memory) at \mi{c}{data_out}.
	\item Increasing the address shows the following instructions.
	\item Setting \mi{c}{data_in} and the load signal to high shows that the data at \mi{c}{0x2} is overwritten.
\end{itemize}

\newpage
\subsection{\mi{c}{control_memory}}
\begin{figure}[h!]
	\centering
	\includegraphics[width=\textwidth]{control_memory_tb}
	\caption{\mi{c}{control_memory} testbench results}
	\label{fig:cmemory}
\end{figure}

Figure~\ref{fig:cmemory} shows the simulation results of the 256x 28-bit word control (microcode) memory.

This memory just outputs the control word at the input address.

\newpage
\subsection{\mi{c}{flag_mux}}
\begin{figure}[h!]
	\centering
	\includegraphics[width=\textwidth]{flag_mux_tb}
	\caption{\mi{c}{flag_mux} testbench results}
	\label{fig:fmux}
\end{figure}

Figure~\ref{fig:fmux} shows the simulation results of the flags mux, used to determine if the next CAR address 
should be loaded or incremented.

\mi{c}{flag} is the value of the bit selected from the internal \mi{c}{flags} vector by \mi{c}{s}.
\begin{itemize}
	\item \mi{vhdl}{flags(7) <= ~Z}
	\item \mi{vhdl}{flags(6) <= ~C}
	\item \mi{vhdl}{flags(5 downto 2) <= flags_in} (in the order \mi{c}{NZVC})
	\item \mi{vhdl}{flags(1) <= "1"} (constant)
	\item \mi{vhdl}{flags(0) <= "0"} (constant)
\end{itemize}

\newpage
\subsection{\mi{c}{car}}
\begin{figure}[h!]
	\centering
	\includegraphics[width=\textwidth]{car_tb}
	\caption{\mi{c}{car} testbench results}
	\label{fig:car}
\end{figure}

Figure~\ref{fig:car} shows the simulation results of the Control Address Register (CAR).

\begin{itemize}
	\item When \mi{c}{load} is low, the CAR will increment its value by 1 on every clock.
		When high, it will load the input address on the next clock.
	\item Pulling \mi{c}{reset} high will set the address to \mi{c}{0xc0}, the address of the instruction fetch
		control word in the control memory.
\end{itemize}

\newpage
\subsection{\mi{c}{pc_extend}}
\begin{figure}[h!]
	\centering
	\includegraphics[width=\textwidth]{pc_extend_tb}
	\caption{\mi{c}{pc_extend} testbench results}
	\label{fig:pcext}
\end{figure}

Figure~\ref{fig:pcext} shows the simulation results of the Program Counter load extender.
This simply takes a 6 bit input (from the instruction register) and extends the top bit (sign) to 16 bits for 
input into the PC.

\newpage
\subsection{\mi{c}{pc}}
\begin{figure}[h!]
	\centering
	\includegraphics[width=\textwidth]{pc_tb}
	\caption{\mi{c}{pc} testbench results}
	\label{fig:pc}
\end{figure}

Figure~\ref{fig:pc} shows the simulation results of the Program Counter (PC).
\begin{itemize}
	\item When \mi{c}{inc} is high, the PC will increment its value on every clock.
	\item When \mi{c}{load} is high, the PC will add the input offset to its current value on the next clock.
\end{itemize}

\newpage
\section{Final processor}
This section will discuss the implemented instructions and results of the test program which utilises them.
\begin{figure}[h!]
	\centering
	\includegraphics[width=\textwidth]{computer_overview}
	\caption{Complete processor testbench}
	\label{fig:coverview}
\end{figure}

\newpage
\subsection{Implemented instructions}
Below is a table of all of the implemented instructions and their corresponding opcodes and microcode words.
\begin{center}
\begin{tabular}{|l|l|l||l|}
	\hline
	\textbf{Opcode} & \textbf{Micro-op} & \textbf{Name} & \textbf{Description} \\
	\hline

	0   & \mi{vhdl}{x"0020000"} & \mi{c}{HLT} & halt (loop forever) \\
	1   & \mi{vhdl}{x"c020224"} & \mi{c}{ADI} & add immediate \\
	2   & \mi{vhdl}{x"c020254"} & \mi{c}{SUI} & subtract immediate \\
	3   & \mi{vhdl}{x"c020014"} & \mi{c}{INC} & increment register \\
	4   & \mi{vhdl}{x"c020064"} & \mi{c}{DEC} & decrement register \\
	5   & \mi{vhdl}{x"c020024"} & \mi{c}{ADD} & add registers \\
	6   & \mi{vhdl}{x"c020054"} & \mi{c}{SUB} & subtract registers \\
	7   & \mi{vhdl}{x"c020084"} & \mi{c}{AND} & logical \mi{c}{&} \\
	8   & \mi{vhdl}{x"c0200a4"} & \mi{c}{OR}  & logical \mi{c}{|} \\
	9   & \mi{vhdl}{x"c0200c4"} & \mi{c}{XOR} & logical \mi{c}{^} \\
	10  & \mi{vhdl}{x"c0200e4"} & \mi{c}{NOT} & logical \mi{c}{~} \\
	11  & \mi{vhdl}{x"c02000c"} & \mi{c}{LD}  & load from memory \\
	12  & \mi{vhdl}{x"c020001"} & \mi{c}{ST}  & store to memory \\
	13  & \mi{vhdl}{x"c020201"} & \mi{c}{STI} & store immediate to memory \\
	14  & \mi{vhdl}{x"8121004"} & \mi{c}{SLI} & shift left by immediate (stores source in r8), dst and a must be the same! \\
	15  & \mi{vhdl}{x"8021004"} & \mi{c}{SL}  & shift left by register (stores source in r8), dst and a must be the same! \\
	16  & \mi{vhdl}{x"8821004"} & \mi{c}{SRI} & shift right by immediate (stores source in r8), dst and a must be the same! \\
	17  & \mi{vhdl}{x"8721004"} & \mi{c}{SR}  & shift right by register (stores source in r8), dst and a must be the same! \\
	18  & \mi{vhdl}{x"c022000"} & \mi{c}{B}   & unconditional branch \\
	19  & \mi{vhdl}{x"8e20000"} & \mi{c}{BEQ} & branch if register is zero ("equal") \\
	127 & \mi{vhdl}{x"c020000"} & \mi{c}{NOP} & do nothing \\
	\hline
\end{tabular}
\end{center}

\smallskip
In order to fetch and execute the next instruction, there are two additional micro-ops towards the end of the 
control memory:

\begin{minted}[tabsize=4]{vhdl}
	192 =>		x"000c002", -- instruction fetch (and increment pc)
	193 =>		x"0030000", -- instruction execute
\end{minted}

\smallskip
Any locations in control memory not otherwise set are filled with \mi{vhdl}{x"002000"} (\mi{c}{HLT}) so that undefined opcodes stop the processor immediately:
\begin{minted}[tabsize=4]{vhdl}
	others =>	x"0020000"  -- go to halt instruction if unknown opcode
\end{minted}

\newpage
\smallskip
A few of the instructions (left / right shift and conditional branching) require additional microcode (note the next address portion of the control words above jumping to this additional code):
\begin{minted}[tabsize=4]{vhdl}
	-- left shift
	128 =>		x"8220104", -- copy the shift amount into dst (from register)
	129 =>		x"0000304", -- copy the shift amount into dst (from immediate)
	130 =>		x"8680000", -- is the shift amount (in a) equal to 0? if so, go to setting dst to r8. otherwise, continue.
	131 =>		x"0001584", -- shift r8 left by one
	132 =>		x"0000064", -- decrement a
	133 =>		x"8220000", -- goto checking shift done
	134 =>		x"c020804", -- set dst to r8 and goto IF

	-- right shift
	135 =>		x"8920104", -- copy the shift amount into dst (from register)
	136 =>		x"0000304", -- copy the shift amount into dst (from immediate)
	137 =>		x"8d80000", -- is the shift amount (in a) equal to 0? if so, go to setting dst to r8. otherwise, continue.
	138 =>		x"0001544", -- shift r8 right by one
	139 =>		x"0000064", -- decrement a
	140 =>		x"8920000", -- goto checking shift done
	141 =>		x"c020804", -- set dst to r8 and goto IF

	-- beq
	142 =>		x"1280000", -- goto unconditional branch if zero
	143 =>		x"c020000", -- otherwise continue as normal
\end{minted}


\newpage
\subsection{Test program}
Below is the sample program as initialized into the start of memory.
\begin{minted}{vhdl}
x"0203", -- adi r0, r0, 3
x"0242", -- adi r1, r0, 2
x"048c", -- sui r2, r1, 4
x"06d0", -- inc r3, r2
x"0918", -- dec r4, r3
x"0b59", -- add r5, r3, r1
x"0daa", -- sub r6, r5, r2
x"0fdd", -- and r7, r3, r5 ; (2 & 7)
x"11c8", -- or r7, r1, r0  ; (5 | 3)
x"13f5", -- xor r7, r6, r5 ; (6 ^ 7)
x"15c8", -- not r7, r1     ; (~5)

x"13b6", -- xor r6, r6, r6
x"17b0", -- ld r6, [r6]
x"036f", -- adi r5, r5, #7
x"036f", -- adi r5, r5, #7
x"036f", -- adi r5, r5, #7
x"036f", -- adi r5, r5, #7
x"036f", -- adi r5, r5, #7
x"182f", -- st r7, [r5] ; (r5 = 42)
x"17a8", -- ld r6, [r5]
x"1a2b", -- sti 3, [r5]
x"17a8", -- ld r6, [r5]

x"1db5", -- sli r6, 5
x"1f6b", -- sl r5, r3 ; (<< 2)
x"1db0", -- sli r6, 0
x"1292", -- xor r2, r2, r2
x"1f6a", -- sl r5, r2 ; (<< 0)

x"21fa", -- sri r7, 2
x"2368", -- sr r5, r0 ; (>> 3)

x"2403", -- b +3
x"13ff", -- xor r7, r7, r7
x"2402", -- b +2
x"03ff", -- adi r7, r7, 7
x"25c5", -- b -3

x"260a", -- beq +2, r1
x"0449", -- sui r1, r1, 1
x"25c5", -- b -3
x"024f", -- adi r1, r1, 7
x"fe00", -- nop
x"fe00", -- nop

x"0000"  -- hlt
\end{minted}

\end{document}
# vim: nofoldenable
